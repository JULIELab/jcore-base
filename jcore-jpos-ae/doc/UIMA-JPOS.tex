%\documentclass[11pt]{article}
\documentclass[11pt,a4paper,halfparskip]{scrartcl}
%\usepackage[pdftex]{graphics} 
\usepackage{graphicx}
\usepackage[utf8x]{inputenc}
\usepackage{url} 
\usepackage[T1]{fontenc}
\usepackage{ucs} 

%\setkomafont{sectioning}{\bfseries}
\pagestyle{plain}
%\typearea{10}

\title{\small{Documentation for}\\\huge UIMA Wrapper for JULIE Lab Part of Speech Tagger\\\vspace{3mm}\small{Version 1.0}}

\author{\normalsize Johannes Hellrich\\
  \normalsize  Jena University Language \& Information Engineering (JULIE) Lab\\
  \normalsize F\"urstengraben 30 \\
  \normalsize D-07743 Jena, Germany\\
  {\normalsize \tt johannes.hellrich@uni-jena.de} }

\date{}

\begin{document}
\maketitle

\section{Objective}

The UIMA Wrapper for JULIE Lab Named Entity Tagger (UIMA-JPOS) is an
UIMA wrapper for the JULIE Lab Part of Speech Tagger (JPOS).  It is part
of the JULIE Lab NLP tool suite\footnote{\url{http://www.julielab.de/}}
which contains several NLP components (all UIMA compliant) from
sentence splitting to named entity recognition and normalization as
well as a comprehensive UIMA type system.

For annotating tokens with their part of speech, this
analysis engine employs the JULIE Lab Part of Speech Tagger (JPOS). JPOS uses a machine learning (ML) approach, generating (ML-)features in order to select POS tags for a
given text of written natural language.  JPOS offers the possibility
to configure the feature generation. As JPOS needs a UIMA pipeline providing sentence and token annotations in its CAS. It then modifies the token annotations by adding POS tags.


\section{About this documentation}
This is a documentation on using the UIMA-compliant version of
JPOS. UIMA-JPOS is a wrapper to JPOS, which actually does all the
named entity recognition. To get more information on JPOS itself,
please refer to its documentation.


\section{Changelog}
\begin{description}
\item[1.0] Initial release.
\end{description}


\section{Requirements and Dependencies}

UIMA-JPOS is completely written in Java using Apache UIMA \footnote{\url{https://uima.apache.org}}. It
requires Java 1.7 (or above).

The input and output of an AE takes place by annotation objects. The
classes corresponding to these objects are part of a \emph{JULIE Lab
  UIMA Type System}.\footnote{The \emph{JULIE UIMA type system} can be
  obtained separately from \url{http://www.julielab.de/}. However, the
  necessary parts of the type system are already contained in this
  package.} When refering to UIMA annotation types we mean types from
the JULIE Lab UIMA type system. 

\section{Using the AE -- Descriptor Configuration}

In UIMA, each component is configured by a descriptor in XML. In the
following we describe how the descriptor required by this AE can be
created (or modified) with the \emph{Component Descriptor Editor}, an
Eclipse plugin which is part of the UIMA SDK.

A descriptor contains information on different aspects. The following
subsection refers to each sub aspect of the descriptor which is, in
the Component Descriptor Editor, a separate \emph{tabbed page}. For an
indepth description of the respective configuration aspects or tabs,
please refer to the \emph{UIMA SKD User's
  Guide}, especially
the chapter on ``Component Descriptor Editor User's Guide''.

To define your descriptor go through each tabbed pages mentioned here,
make your respective entries (especially in page \emph{Parameter
  Settings} you will be able to configure UIMA-JPOS to your needs) and
save the descriptor as \url{SomeName.xml}.

\paragraph{Overview}
This tab provides general informtion about the component. For the UIMA-JPOS
you need to provide the information as specified in Table
\ref{tab:overview}.

\begin{table}[h!]
  \centering
  \begin{tabular}{|p{4cm}|p{4cm}|p{5cm}|}
    \hline
    Subsection & Key & Value \\
    \hline\hline
    Implementation Details & Implementation Language &  Java\\
    \cline{2-3}
    & Engine Type & primitive\\
    \hline
    Runtime Information & updates the CAS & check \\
    \cline{2-3}
    & multiple deployment allowed & check\\
    \cline{2-3}
    & outputs new CASes & don't check \\
    \cline{2-3}
    & Name of the Java class file & \url{de.julielab.jules.ae.postagger.POSAnnotator}\\
    \hline
    Overall Identification Information & Name & UIMA-JPOS \\
    \cline{2-3}
    & Version &  1.0\\
    \cline{2-3}
    & Vendor & JULIE Lab\\
    \cline{2-3}
    & Description & you may keep this empty\\
    \hline
  \end{tabular}
  \caption{Overview/General Settings for AE.}
  \label{tab:overview}
\end{table}


\paragraph{Aggregate}
Not needed here, as this AE is a primitive.

\paragraph{Parameters}
\label{sss:parameters}

See Table \ref{tab:parameters} for a specification of the
configuration parameters of this AE. Do not check ``Use Parameter
Groups'' in this tab.

\begin{table}[h!]
  \centering
  \begin{tabular}{|p{4cm}|p{2cm}|p{2cm}|p{2cm}|p{4cm}|}
    \hline
    Parameter Name & Parameter Type & Mandatory & Multivalued & Description \\
    \hline
    ModelFilename & String & yes & no & specifies which model JPOS
    should use\\ 
    \hline
    tagset & String & yes & no & specifies which POS tag set to use\\
    \hline\hline
  \end{tabular}
  \caption{Parameters of this AE.}
  \label{tab:parameters}
\end{table}

\paragraph{Parameter Settings}
\label{sss:param_settings}

The specific parameter settings are filled in here. For each of the
parameters defined in \ref{sss:parameters}, add the respective values
here (has to be done at least for each parameter that is defined as
mandatory). See Table \ref{tab:param_settings} for the respective
parameter settings of this AE.

\begin{table}[h!]
  \centering
  \begin{tabular}{|p{4cm}|p{4cm}|p{8cm}|}
    \hline
    Parameter Name & Parameter Syntax & Example \\
    \hline\hline
    ModelFilename & Give either the complete path to the model file
    & \url{/path/to/model} or only its \textbf{name if it resides in the classpath}\\
       \hline
    tagset & full name of the POS tag set used
    & \url{some.pos.TagSet} \\
    \hline\hline
  \end{tabular}
  \caption{Parameter settings of this AE.}
  \label{tab:param_settings}
\end{table}

\paragraph{Type System}
\label{sss:type_system}
On this page, go to \emph{Imported Type} and import \texttt{"julie-all-types"} by name.

\paragraph{Capabilities}
\label{sss:capabilities}
UIMA-JPOS needs as input annotations from type
\url{de.julielab.jules.types.Sentence} and
\url{de.julielab.jules.types.Token}. It modifies the annotations from type
\url{de.julielab.jules.types.Token}. 
% adapt if needed
\begin{table}[h!]
  \centering
  \begin{tabular}{|p{5cm}|p{2cm}|p{2cm}|}
    \hline
    Type & Input & Output \\
    \hline\hline
     de.julielab.jules.types.Sentence &  $\surd$ & \\
      \hline
     de.julielab.jules.types.Token &  $\surd$ & $\surd$ \\
      \hline
  \end{tabular}
  \caption{Capabilities of this AE.}
  \label{tab:capabilities}
\end{table} 



\paragraph{Index}

Nothing needs to be done here.

\paragraph{Resources}

Nothing needs to be done here.


\section{Modifying the Descriptors}

This PEAR package contain one descriptor for UIMA-JPOS configured for tagging German biomedical texts. We also provide a model for German newspaper text. You can train other models using the JPOS command-line tool; usig JPOS for English is not advised.

\section{Copyright and License}

This software is Copyright (C) 2015 Jena University Language \& Information
Engineering Lab (Friedrich-Schiller University Jena, Germany), and is
licensed under the terms of the Common Public License, Version 1.0 or (at
your option) any subsequent version.

The license is approved by the Open Source Initiative, and is
available from their website at \url{http://www.opensource.org}.

\bibliographystyle{alpha}
\bibliography{literature}


\end{document}