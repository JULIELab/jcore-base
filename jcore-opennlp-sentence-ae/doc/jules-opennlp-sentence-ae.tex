\documentclass[11pt,a4paper,halfparskip]{scrartcl}
\usepackage{graphicx}
\usepackage[utf8x]{inputenc}
\usepackage{url} 
\usepackage[T1]{fontenc}
\usepackage{ucs} 
\pagestyle{plain}

\title{\small{Documentation for}\\\huge JULIE Lab UIMA Wrapper for \\OpenNLP Sentence Splitter}

\author{\normalsize Ekaterina Buyko\\
  \normalsize  Jena University Language \& Information Engineering (JULIE) Lab\\
  \normalsize F\"urstengraben 30 \\
  \normalsize D-07743 Jena, Germany\\
  {\normalsize \tt \{buyko\}@coling-uni-jena.de} }


\date{}

\begin{document}
\maketitle

\section{Objective}


The \textsc{OpenNLP Sentence Splitter}\footnote{http://www.opennlp.org} identifies sentences in 
a text with respect to user-defined end-of-sentence (\textit{eos}) punctuation
markers.\footnote{With regard to the biological domain we extended the default set $\lbrace . ! ? 
\rbrace$ with $\lbrace : ] \rbrace$.  In its default mode, the
\textsc{OpenNLP Sentence Splitter} only considers words containing these
\textit{eos}.} \textsc{Julie Lab UIMA Wrapper for OpenNLP Sentence Splitter} is part of the Julie
Lab NLP tool suite\footnote{\url{http://www.julielab.de/}} which contains several NLP components
(all
UIMA compliant) from sentence splitting to named entity recognition and normalization as well as a
comprehensive UIMA type system.

\textsc{UIMA Wrapper for OpenNLP Sentence Splitter} is currently available in version 1.1.  For more
detailed information on the functioning of the \textsc{OpenNLP Sentence Splitter} check
\url{http://www.opennlp.org}.


% now you can put a longer description of the tool here


%\section{Changelog}
% uncomment when needed


\section{Requirements and Dependencies}

% mostly our tools will be based on java 1.5 and use UIMA
\textsc{Julie Lab UIMA Wrapper for OpenNLP Sentence Splitter} is written in Java 1.5 using Apache
UIMA version 2.2.0-incubation\footnote{\url{http://incubator.apache.org/uima/}}. It
was not tested with other UIMA versions.

% ref to our type system
The input and output of an AE takes place by annotation objects. The
classes corresponding to these objects are part of a \emph{Julie Lab UIMA
  type systems}\footnote{The \emph{Julie Lab UIMA type systems} can be
  obtained from \url{http://www.julielab.de/}}.

The wrapper comes as a UIMA pear file. Run the Pear-Installer (e.g.,
\url{./runPearInstaller.sh} for Linux) from your UIMA-bin directory.
After installation, you will find a subfolder \url{desc} in you
installation folder. This directory contains a descriptor
\url{SentenceAnnotator.xml}. You may now e.g. run UIMA's
Collection Proeccessing Engine Configurator (\url{cpeGUI.sh}) and add
the wrapper as a component into your NLP pipeline.

This pear package also contains a model for sentence splitting. The
model was trained on a bio-medical corpus \textsc{GENIA} (\cite{ohta02}).
An accuracy of 99.0\% is yielded on this data using 10-fold cross-validation.  
You will find the model trained on this data in the directory \url{resources}.
% now other dependencies you tool might have (you might organize this
% a subsections as well)




\section{Using the AE -- Descriptor Configuration}
% carefully edit this section!

In UIMA, each component is configured by a descriptor in XML. In the
following we describe how the descriptor required by this AE can be
created with \emph{Component Descriptor Editor}, an Eclipse plugin
which is part of the UIMA SDK.

A descriptor contains information on different aspects. The following
subsection refers to each sub aspect of the descriptor which is, in
the Component Descriptor Editor, a separate \emph{tabbed page}. For an
indepth description of the respective configuration aspects or tabs,
please refer to the \emph{UIMA SKD User's
  Guide}\footnote{\url{http://incubator.apache.org/uima/}}, especially
chapter 12 on ``Component Descriptor Editor User's Guide''.

To define your descriptor go through each tabbed pages mentioned
here, make your respective entries (especially in page \emph{Parameter
Settings} you will be able to configure \textsc{OpenNLP Sentence Splitter} 
to your needs) and save the descriptor as \\ \url{SentenceAnnotator.xml}.

\paragraph{Overview}
This tab provides general informtion about the component. For the
OpenNLP Sentence Splitter you need to provide the information as
specified in Table
\ref{tab:overview}.
% adapt to your needs, remember to change values in tabular below!

\begin{table}[h!]
  \centering
  \begin{tabular}{|p{3.5cm}|p{4cm}|p{6cm}|}
    \hline
    Subsection & Key & Value \\
    \hline\hline
    Implementation Details & Implementation Language &  \textsc{Java} \\
    \cline{2-3}
    & Engine Type & Primitive\\
    \hline
    Runtime Information & updates the CAS & yes  \\
    \cline{2-3}
    & multiple deployment allowed & yes \\
    \cline{2-3}
    & outputs new CASes & no \\
    \cline{2-3}
    & Name of the Java class file & \url{de.julielab.jules.ae.opennlp.SentenceAnnotator}\\
    \hline
    Overall Identification Information & Name & \textsc{JULIELAB UIMA Wrapper for OpenNLP Sentence
 Splitter} \\
    \cline{2-3}
    & Version & 2.1 \\
    \cline{2-3}
    & Vendor & julielab\\
    \cline{2-3}
    & Description & see above\\
    \hline
  \end{tabular}
  \caption{Overview/General Settings for AE.}
  \label{tab:overview}
\end{table}


\paragraph{Aggregate}
% for primitive AEs this does not have to be set
Not needed here, as this AE is a primitive.

\paragraph{Parameters}
\label{sss:parameters}
% adapt to your needs

See Table \ref{tab:parameters} for a specification of the
configuration parameters of this AE. Do not check ``Use Parameter
Groups'' in this tab.

\begin{table}[h!]
  \centering
  \begin{tabular}{|p{4cm}|p{2cm}|p{2cm}|p{2cm}|p{4cm}|}
    \hline 
    Parameter Name & Parameter Type & Mandatory & Multivalued & Description \\
   \hline \hline
      modelFile & String & yes &no &  path to the \textsc{OpenNLP Sentence Splitter}
model\\
    % add your parameters here
  \hline
  \end{tabular}
  \caption{Parameters of this AE.}
  \label{tab:parameters}
\end{table}


\paragraph{Parameter Settings}
\label{sss:param_settings}
% adapt to your needs

The specific parameter settings are filled in here. For each of the
parameters defined in \ref{sss:parameters}, add the respective values
here (has to be done at least for each parameter that is defined as
mandatory). See Table \ref{tab:param_settings} for the respective
parameter settings of this AE.

\begin{table}[h!]
  \centering
  \begin{tabular}{|p{4cm}|p{4cm}|p{7cm}|}
    \hline
    Parameter Name & Parameter Syntax & Example \\
    \hline\hline
    modelFile & model.bin.gz & resources/SentDetectGenia.bin.gz\\
  \hline
  \end{tabular}
  \caption{Parameter settings of this AE.}
  \label{tab:param_settings}
\end{table}

\paragraph{Type System}
\label{sss:type_system}
On this page, go to \emph{Imported Type} and add the \emph{julie-morpho-syntax-types.xml} type system.. (Use ``Import by Location'').


\paragraph{Capabilities}
% adapt if needed
Nothing needs to be done here.


\paragraph{Index}
% adapt if needed
Nothing needs to be done here.

\paragraph{Resources}
% adapt if needed
Nothing needs to be done here.


\section{Copyright and License}
% leave unchanged
This software is Copyright (C) 2007 Jena University Language \& Information
Engineering Lab (Friedrich-Schiller University Jena, Germany), and is
licensed under the terms of the Common Public License, Version 2.1.

The license is approved by the Open Source Initiative, and is
available from their website at \url{http://www.opensource.org}.

\bibliographystyle{alpha}
\bibliography{/home/buyko/coling/papers/biblio/paper}


\end{document}
