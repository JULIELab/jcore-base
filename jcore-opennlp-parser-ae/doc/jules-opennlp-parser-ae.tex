\documentclass[11pt,a4paper,halfparskip]{scrartcl}
\usepackage{graphicx}
\usepackage[utf8x]{inputenc}
\usepackage{url}
\usepackage{helvet}
\usepackage{times}
\usepackage[T1]{fontenc}
\usepackage{ucs}
\usepackage{wasysym}
\pagestyle{plain}


\title{\small{Documentation for}\\\huge JULIE Lab UIMA Wrapper for \\OpenNLP Parser}

\author{\normalsize Ekaterina Buyko\\
  \normalsize  Jena University Language \& Information Engineering (JULIE) Lab\\
  \normalsize F\"urstengraben 30 \\
  \normalsize D-07743 Jena, Germany\\
  {\normalsize \tt \{buyko\}@coling-uni-jena.de} }


\date{}

\begin{document}
\maketitle

\section{Objective}




\textsc{Julie Lab UIMA Wrapper for OpenNLP Parser} is part of the Julie Lab NLP tool
suite\footnote{\url{http://www.julielab.de/}} which contains several NLP components 
(all UIMA compliant) from sentence splitting to named entity recognition and 
normalization as well as a comprehensive UIMA type system. The \textsc{OpenNLP Parser}\footnote{\url{http://www.opennlp.org}} 
provides consituency-based parsing. For more detailed information on the 
functioning of the \textsc{OpenNLP Parser} check \url{http://www.opennlp.org}.

\textsc{UIMA Wrapper for OpenNLP Parser} is currently available in version 2.1.


% now you can put a longer description of the tool here


%\section{Changelog}
% uncomment when needed


\section{Requirements and Dependencies}

% mostly our tools will be based on java 1.5 and use UIMA


\textsc{Julie Lab UIMA Wrapper for OpenNLP Parser} is written in Java 1.6 using Apache
UIMA version 2.1.0-incubation\footnote{\url{http://incubator.apache.org/uima/}}. It
was not tested with other UIMA versions.

% ref to our type system
The input and output of an AE takes place by annotation objects. The 
classes corresponding to these objects are part of a \emph{Julie Lab UIMA
  type systems}\footnote{The \emph{Julie Lab UIMA type systems} can be
  obtained from \url{http://www.julielab.de/}}.

The wrapper comes as a UIMA pear file. Run the Pear-Installer (e.g.,
\url{./runPearInstaller.sh} for Linux) from your UIMA-bin directory.
After installation, you will find a subfolder \url{desc} in your
installation folder. This directory contains a descriptor
\url{ParseAnnotator.xml}. You may now e.g. run UIMA's
Collection Proeccessing Engine Configurator (\url{cpeGUI.sh}) and add
the wrapper as a component into your NLP pipeline.

This pear package also contains models for parsing. The
models are trained on two bio-medical corpora respectively: \textsc{GENIA} (\cite{ohta02})
and PennBIoIE\footnote{\url{http://bioie.ldc.upenn.edu/}}. An accuracy of 86.1\% is
yielded on the \textsc{GENIA} using 10-fold cross-validation \cite{buyko.biolink06}. You will find the models in the
directory \url{resources}.
% now other dependencies you tool might have (you might organize this
% a subsections as well)




\section{Using the AE -- Descriptor Configuration}
% carefully edit this section!

In UIMA, each component is configured by a descriptor in XML. In the
following we describe how the descriptor required by this AE can be
created with \emph{Component Descriptor Editor}, an Eclipse plugin
which is part of the UIMA SDK.

A descriptor contains information on different aspects. The following
subsection refers to each sub aspect of the descriptor which is, in
the Component Descriptor Editor, a separate \emph{tabbed page}. For an
indepth description of the respective configuration aspects or tabs,
please refer to the \emph{UIMA SDK User's
  Guide}\footnote{\url{http://incubator.apache.org/uima/}}, especially
chapter 12 on ``Component Descriptor Editor User's Guide''.

To define your descriptor go through each tabbed pages mentioned
here, make your respective entries (especially in page \emph{Parameter
Settings} you will be able to configure \textsc{OpenNLP Parser} 
to your needs) and save the descriptor as \url{DescriptorName.xml}.

\paragraph{Overview}
This tab provides general informtion about the component. For the
OpenNLP Parser you need to provide the information as
specified in Table
\ref{tab:overview}.
% adapt to your needs, remember to change values in tabular below!

\begin{table}[h!]
  \centering
  \begin{tabular}{|p{4cm}|p{4cm}|p{6cm}|}
    \hline
    Subsection & Key & Value \\
    \hline\hline
    Implementation Details & Implementation Language &  \textsc{Java} \\
    \cline{2-3}
    & Engine Type & Primitive\\
    \hline
    Runtime Information & updates the CAS & yes  \\
    \cline{2-3}
    & multiple deployment allowed & yes \\
    \cline{2-3}
    & outputs new CASes & no \\
    \cline{2-3}
    & Name of the Java class file & \url{de.julielab.jules.ae.opennlp.ParseAnnotator}\\
    \hline
    Overall Identification Information & Name & \textsc{jules-opennlp-parser-ae}
\\
    \cline{2-3}
    & Version & 2.1 \\
    \cline{2-3}
    & Vendor & julielab\\
    \cline{2-3}
    & Description & see above\\
    \hline
  \end{tabular}
  \caption{Overview/General Settings for AE.}
  \label{tab:overview}
\end{table}


\paragraph{Aggregate}
% for primitive AEs this does not have to be set
Not needed here, as this AE is a primitive.

\paragraph{Parameters}
\label{sss:parameters}
% adapt to your needs

See Table \ref{tab:parameters} for a specification of the
configuration parameters of this AE. Do not check ``Use Parameter
Groups'' in this tab.

\begin{table}[h!]
  \centering
  \begin{tabular}{|p{4cm}|p{2cm}|p{2cm}|p{2cm}|p{4cm}|}
    \hline 
    Parameter Name & Parameter Type & Mandatory & Multivalued & Description \\
   \hline \hline
	modelDir & String & yes &no & Path to the directory with \textsc{OpenNLP Parser} models\\
	tagset & String & sey & no & Cas type to annotate (see \emph{JULIE UIMA type system)}\\
	useTagdict & Boolean & yes & no & \url{true} if a tag dictionary should be used\\
	caseSensitive & Boolean & no & no & \url{true} if a tag dictionary is case senstive\\
	mappings & String (multi-valued) & yes & yes & Mappings between constituent categories
provided by the \textsc{OpenNLP Parser} and \emph{Constituent} feature \emph{cat} (see
\emph{jules-morpho-syntpes.xml})\\
	beamSize & Integer& no & no & Beam size\\
	advancePercentage & String & no & no & Amount of probability mass required of advanced outcomes\\
	fun & Boolean & no & no & \url{true} if parsing with functional tags (e.g. subj, obj)\\

    % add your parameters here
  \hline
  \end{tabular}
  \caption{Parameters of this AE.}
  \label{tab:parameters}
\end{table}


\paragraph{Parameter Settings}
\label{sss:param_settings}
% adapt to your needs

The specific parameter settings are filled in here. For each of the
parameters defined in \ref{sss:parameters}, add the respective values
here (has to be done at least for each parameter that is defined as
mandatory). See Table \ref{tab:param_settings} for the respective
parameter settings of this AE.

\begin{table}[h!]
  \centering
  \begin{tabular}{|p{3cm}|p{5cm}|p{7cm}|}
    \hline
    Parameter Name & Parameter Syntax & Example \\
    \hline\hline
    	modelDir & String & resources/modelsGenia\\
	tagset & CAS sub-type of \emph{Constituent} & de.julielab.jules.types.GENIAConstituent\\
	useTagdict & Boolean & \url{true}\\
	mappings & OpenNLP Name;CAS Name & S;S\\
  \hline
  \end{tabular}
  \caption{Parameter settings of this AE.}
  \label{tab:param_settings}
\end{table}

\paragraph{Type System}
\label{sss:type_system}
On this page, go to \emph{Imported Type} and add the \emph{julie-morpho-syntax-types.xml} type
system. (Use ``Import by Location'').

\paragraph{Capabilities}
% adapt if needed
See Table \ref{tab:capabilities}

\begin{table}[h!]
  \centering
  \begin{tabular}{|p{6cm}|p{2cm}|p{2cm}|}
    \hline
    Type & Input & Output \\
    \hline\hline
    	de.julielab.jules.types.Sentence & \checked & \\
	de.julielab.jules.types.Token & \checked & \\
	de.julielab.jules.types.Constituent & & \checked\\
  \hline
  \end{tabular}
  \caption{Capabilities of this AE.}
  \label{tab:capabilities}
\end{table}


\paragraph{Index}
% adapt if needed
Nothing needs to be done here.

\paragraph{Resources}
% adapt if needed
Nothing needs to be done here.


\section{Copyright and License}
% leave unchanged
This software is Copyright (C) 2007 Jena University Language \& Information
Engineering Lab (Friedrich-Schiller University Jena, Germany), and is
licensed under the terms of the Common Public License, Version 1.0 or (at
your option) any subsequent version.

The license is approved by the Open Source Initiative, and is
available from their website at \url{http://www.opensource.org}.

\bibliographystyle{alpha}
\bibliography{/home/buyko/coling/papers/biblio/paper}


\end{document}
