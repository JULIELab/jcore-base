\documentclass[11pt,a4paper,halfparskip]{scrartcl}
\usepackage{graphicx}
\usepackage[utf8x]{inputenc}
\usepackage{url}
\usepackage{helvet}
\usepackage{times}
\usepackage[T1]{fontenc}
\usepackage{ucs}
\usepackage{wasysym}
\pagestyle{plain}


\title{\small{Documentation for}\\\huge JULIE Lab ACE Reader}

\author{\normalsize Ekaterina Buyko ~~~~~~~~~~~~~~~ Oleg Lichtenwald\\
  \normalsize  Jena University Language \& Information Engineering (JULIE) Lab\\
  \normalsize F\"urstengraben 30 \\
  \normalsize D-07743 Jena, Germany\\
  {\normalsize \tt \{buyko, lichtenwald\}@coling-uni-jena.de} }


\date{}

\begin{document}
\maketitle

\section{Objective}




The \textsc{JULIE Lab ACE Reader} is an UIMA Collection Reader (CR). It reads the English section of the ACE 2005 Multilingual Training
Corpus data, which is given as XML files, and converts it to types defined in the UIMA type system that we provide as well. 
The \textsc{JULIE Lab ACE Reader} is part of the JULIE NLP tool
suite\footnote{\url{http://www.julielab.de/}} which contains several NLP components 
(all UIMA compliant) from sentence splitting to named entity recognition and 
normalization as well as a comprehensive UIMA type system.

The \textsc{JULIE Lab ACE Reader} is currently available in version 2.0.1. For more detailed information about the ACE data, please read \cite{doddington04}.


% now you can put a longer description of the tool here


%\section{Changelog}
% uncomment when needed


\section{Requirements and Dependencies}

% mostly our tools will be based on java 1.5 and use UIMA


The \textsc{JULIE Lab ACE Reader} is written in Java 5.0 using Apache
UIMA version 2.2.1-incubation\footnote{\url{http://incubator.apache.org/uima/}}. It
was not tested with other UIMA versions.


% ref to our type system
The input of the \textsc{JULIE Lab ACE Reader} can be purchased at the Linguistic Data Consortium (LDC)\footnote{\url{http://www.ldc.upenn.edu/}}. The Output of the CR takes place by annotation objects. The 
classes corresponding to these objects are part of a \emph{JULIE UIMA
  Type System}\footnote{The \emph{JULIE UIMA type system} can be
  obtained from \url{http://www.julielab.de/}}.

The CR comes as a UIMA pear file. Run the Pear-Installer (e.g.,
\url{./runPearInstaller.sh} for Linux) from your UIMA-bin directory.
After installation, you will find a subfolder \url{desc} in your
installation folder. This directory contains a descriptor
\url{ACEReaderDescriptor.xml}. You may now e.g. run UIMA's
Collection Proeccessing Engine Configurator (\url{cpeGUI.sh}) and add
the wrapper as a component into your NLP pipeline.


% now other dependencies you tool might have (you might organize this
% a subsections as well)




\section{Using the CR -- Descriptor Configuration}
% carefully edit this section!

In UIMA, each component is configured by a descriptor in XML. In the
following we describe how the descriptor required by this CR can be
created with \emph{Component Descriptor Editor}, an Eclipse plugin
which is part of the UIMA SDK.

A descriptor contains information on different aspects. The following
subsection refers to each sub aspect of the descriptor which is, in
the Component Descriptor Editor, a separate \emph{tabbed page}. For an
indepth description of the respective configuration aspects or tabs,
please refer to the \emph{UIMA SDK User's
  Guide}\footnote{\url{http://incubator.apache.org/uima/}}, especially
chapter 12 on ``Component Descriptor Editor User's Guide''.

To define your descriptor go through each tabbed pages mentioned
here, make your respective entries (especially in page \emph{Parameter
Settings} you will be able to configure \textsc{JULIE Lab ACE Reader} 
to your needs) and save the descriptor as \url{ACEReaderDescriptor.xml}.

\paragraph{Overview}
This tab provides general informtion about the component. For the 
\textsc{JULIE Lab ACE Reader}  you need to provide the information as
specified in Table
\ref{tab:overview}.
% adapt to your needs, remember to change values in tabular below!

\begin{table}[h!]
  \centering
  \begin{tabular}{|p{4cm}|p{4cm}|p{6cm}|}
    \hline
    Subsection & Key & Value \\
    \hline\hline
    Implementation Details & Implementation Language &  \textsc{Java} \\
    \cline{2-3}
    & Engine Type & Primitive\\
    \hline
    Runtime Information & updates the CAS & yes  \\
    \cline{2-3}
    & multiple deployment allowed & yes \\
    \cline{2-3}
    & outputs new CASes & no \\
    \cline{2-3}
    & Name of the Java class file & \url{de.julielab.jules.reader.AceReader}\\
    \hline
    Overall Identification Information & Name & \textsc{jules-ace-reader}
\\
    \cline{2-3}
    & Version & 2.0.1 \\
    \cline{2-3}
    & Vendor & julielab\\
    \cline{2-3}
    & Description & see above\\
    \hline
  \end{tabular}
  \caption{Overview/General Settings for CR.}
  \label{tab:overview}
\end{table}


\paragraph{Aggregate}
% for primitive AEs this does not have to be set
Not needed here, as this CR is a primitive.

\paragraph{Parameters}
\label{sss:parameters}
% adapt to your needs

See Table \ref{tab:parameters} for a specification of the
configuration parameters of this CR. Do not check ``Use Parameter
Groups'' in this tab.

\begin{table}[h!]
  \centering
  \begin{tabular}{|p{4cm}|p{2cm}|p{2cm}|p{2cm}|p{4cm}|}
    \hline 
    Parameter Name & Parameter Type & Mandatory & Multivalued & Description \\
   \hline \hline
	inputDirectory & String & yes & no & Path to the ACE files\\
	generateJulesTypes & Boolean & no & no & Determines if JULIE Lab Types (julie-semantics-ace-types.xml) should be generated in addition to types from julie-ace-types.xml\\
    % add your parameters here
  \hline
  \end{tabular}
  \caption{Parameters of this CR.}
  \label{tab:parameters}
\end{table}


\paragraph{Parameter Settings}
\label{sss:param_settings}
% adapt to your needs

The specific parameter settings are filled in here. For each of the
parameters defined in \ref{sss:parameters}, add the respective values
here (has to be done at least for each parameter that is defined as
mandatory). See Table \ref{tab:param_settings} for the respective
parameter settings of this CR.

\begin{table}[h!]
  \centering
  \begin{tabular}{|p{3cm}|p{5cm}|p{7cm}|}
    \hline
    Parameter Name & Parameter Syntax & Example \\
    \hline\hline
    	inputDirectory & valid path to the ACE files & resources/AceData\\
	generateJulesTypes & boolean variable & \url{true}\\
  \hline
  \end{tabular}
  \caption{Parameter settings of this CR.}
  \label{tab:param_settings}
\end{table}

\paragraph{Type System}
\label{sss:type_system}
On this page, go to \emph{Imported Type} and add the \emph{JULIE UIMA Type System}. (Use ``Import by Location'').


\paragraph{Capabilities}
% adapt if needed
Nothing needs to be done here.


\paragraph{Index}
% adapt if needed
Nothing needs to be done here.

\paragraph{Resources}
% adapt if needed
Nothing needs to be done here.


\section{Copyright and License}
% leave unchanged
This software is Copyright (C) 2008 Jena University Language \& Information
Engineering Lab (Friedrich-Schiller University Jena, Germany), and is
licensed under the terms of the Common Public License, Version 1.0 or (at
your option) any subsequent version.

The license is approved by the Open Source Initiative, and is
available from their website at \url{http://www.opensource.org}.

\bibliographystyle{alpha}
\bibliography{/home/papers/biblio/paper}


\end{document}
