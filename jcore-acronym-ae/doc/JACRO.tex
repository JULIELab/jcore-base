\documentclass[11pt,a4paper,halfparskip]{scrartcl}
\usepackage{graphicx}
\usepackage[utf8x]{inputenc}
\usepackage{url} 
\usepackage[T1]{fontenc}
\usepackage{ucs} 
\pagestyle{plain}

\title{\small{Documentation for}\\\huge JULIE Lab Acronym
  Annotator\\\vspace{3mm}\small{Version 2.1}}
  

\author{\normalsize Katrin Tomanek ~~~~~~~~~~~~~ Christina Tusche\\
  \normalsize  Jena University Language \& Information Engineering (JULIE) Lab\\
  \normalsize F\"urstengraben 30 \\
  \normalsize D-07743 Jena, Germany\\
  {\normalsize \tt katrin.tomanek@uni-jena.de} }



\date{}

\begin{document}
\maketitle

\section{Objective}

JULIE Lab Acronym Annotator (JACRO) is an UIMA Analysis Engine that
annotates acronyms with their full-forms when locally introduced in
the current document.  It is part of the JULIE Lab NLP tool
suite\footnote{\url{http://www.julielab.de/}} which contains several
UIMA-compliant NLP components from sentence splitting to named entity
recognition and normalization as well as a comprehensive UIMA type
system.

The functionality of the engine is based on the simple algorithm for
abbreviation recognition by Schwartz and
Hearst\footnote{\url{http://biotext.berkeley.edu/papers/psb03.pdf}}. We
have reimplemented the algorithm and extended it with respect to some
pattern definitions and normalizations. 

During the processing of documents, this UIMA annotator takes sentence
annotions from the CAS and creates an \url{Abbreviation} annotation
object for each identified acronym. The \url{Abbreviation} annotation
stores the corresponding full-form, whether the acronym was introduced
at the respective position, and a reference to the full-form in the
text.
%\section{Changelog}
% uncomment when needed


\section{Requirements and Dependencies}

% mostly our tools will be based on java 1.5 and use UIMA
The annotator is completely written in Java (at least Java 1.5
required) using Apache UIMA version
2.2.1-incubation\footnote{\url{http://incubator.apache.org/uima/}}.

% ref to our type system
The input and output of an AE takes place by annotation objects. The
classes corresponding to these objects are part of the \emph{JULIE Lab
  UIMA Type System} in its current version (2.1).\footnote{The
  \emph{JULIE Lab UIMA type system} can be separately obtained from
  \url{http://www.julielab.de/}, however, this package already
  includes the necessary parts of the type system.}

% now other dependencies you tool might have (you might organize this
% a subsections as well)




\section{Using the AE -- Descriptor Configuration}
% carefully edit this section!

In UIMA, each component is configured by a descriptor in XML. In the
following we describe how the descriptor required by this AE can be
created with the \emph{Component Descriptor Editor}, an Eclipse plugin
which is part of the UIMA SDK. 

A descriptor contains information on different aspects. The following
subsection refers to each sub aspect of the descriptor which is, in
the Component Descriptor Editor, a separate \emph{tabbed page}. For an
indepth description of the respective configuration aspects or tabs,
please refer to the \emph{UIMA SKD User's
  Guide}\footnote{\url{http://incubator.apache.org/uima/}}, especially
the chapter on ``Component Descriptor Editor User's Guide''.

To define your descriptor go through each tabbed page mentioned
here, make your respective entries and save
the descriptor as e.g. \url{AcronymAnnotatorDescriptor.xml}.

\textbf{As this package already contains a pre-configured descriptor
  (see \url{desc/AcronymAnnotator.xml}) there is no need to build such
  a descriptor from scratch. However, you might modify the parameter
  settings according to your needs.}

\paragraph{Overview}
This tab provides general informtion about the component. For the
Acronym Annotator you need to provide the information as specified in
Table \ref{tab:overview}.
% adapt to your needs, remember to change values in tabular below!

\begin{table}[h!]
  \centering
  \begin{tabular}{|p{3.5cm}|p{4cm}|p{6cm}|}
    \hline
    Subsection & Key & Value \\
    \hline\hline
    Implementation Details & Implementation Language &  Java\\
    \cline{2-3}
    & Engine Type & primitive\\
    \hline
    Runtime Information & updates the CAS & yes \\
    \cline{2-3}
    & multiple deployment allowed & yes\\
    \cline{2-3}
    & outputs new CASes & yes \\
    \cline{2-3}
    & Name of the Java class file & \url{de.julielab.jules.ae.acronymtagger.AcronymAnnotator}\\
    \hline
    Overall Identification Information & Name & Acronym Annotator \\
    \cline{2-3}
    & Version &  2.1\\
    \cline{2-3}
    & Vendor & julielab\\
    \cline{2-3}
    & Description & you may keep this empty\\
    \hline
  \end{tabular}
  \caption{Overview/General Settings for AE.}
  \label{tab:overview}
\end{table}


\paragraph{Aggregate}
% for primitive AEs this does not have to be set
Not needed here, as this AE is a primitive.

\paragraph{Parameters}
\label{sss:parameters}
% adapt to your needs

See Table \ref{tab:parameters} for a specification of the
configuration parameters of this AE. Do not check ``Use Parameter
Groups'' in this tab.


\begin{table}[ht!]
  \centering
  \begin{tabular}{|p{2.8cm}|p{1.8cm}|p{1.5cm}|p{1.5cm}|p{5cm}|}
    \hline
    Parameter Name & Parameter Type & Mandt. & Multi\-valued & Description \\
    \hline\hline
    ConsistencyAnno & Bool & yes & no & specifies whether only the first or all occurences of the acronym are annotated in the document\\
\hline
MaxLength & Integer &  yes & no & Define how far (how many words, ignoring stopwords) the AE is supposed too look for the beginning of the fullform.\\
    \hline
  \end{tabular}
  \caption{Parameters of this AE.}
  \label{tab:parameters}
\end{table}



\paragraph{Parameter Settings}
\label{sss:param_settings}
% adapt to your needs

The specific parameter settings are filled in here. For each of the
parameters defined in Table \ref{tab:parameters}, add the respective values
here (has to be done at least for each parameter that is defined as
mandatory). See Table \ref{tab:param_settings} for the respective
parameter settings of this AE.

\begin{table}[h!]
  \centering
  \begin{tabular}{|p{2.8cm}|p{5cm}|p{6cm}|}
    \hline
    Parameter Name & Parameter Syntax & Example \\
    \hline\hline
    \hline
    ConsistencyAnno & set to true iff you want to annotate all occurences of the found acronyms &
    true\\
\hline
MaxLength & just an integer & 5 \\
  \hline
  \end{tabular}
  \caption{Parameter settings of this AE.}
  \label{tab:param_settings}
\end{table}

\paragraph{Type System}
\label{sss:type_system}
On this page, go to \emph{Imported Type} and add the following layers
of the \emph{JULIE UIMA Type System} (Use ``Import by Location''):
\url{julie-basic-types.xml} and \url{julie-morpho-syntax-types.xml}.


\paragraph{Capabilities}
\label{sss:capabilities}
JACRO takes as input annotations from type \url{de.julielab.jules.types.Sentence} and returns annotations from type \url{de.julielab.jules.types.Abbreviation}. See Table \ref{tab:capabilities}.
% adapt if needed
\begin{table}[h!]
  \centering
  \begin{tabular}{|p{5cm}|p{2cm}|p{2cm}|}
    \hline
    Type & Input & Output \\
    \hline\hline
     de.julielab.jules.types.Sentence & $\surd$ & \\
      \hline
     de.julielab.jules.types.Abbreviation & &  $\surd$  \\
      \hline
  \end{tabular}
  \caption{Capabilities of this AE.}
  \label{tab:capabilities}
\end{table} 


\paragraph{Index}
% adapt if needed
Nothing needs to be done here.

\paragraph{Resources}
% adapt if needed
Nothing needs to be done here.


\section{Copyright and License}
% leave unchanged
This software is Copyright (C) 2008 Jena University Language \& Information
Engineering Lab (Friedrich-Schiller University Jena, Germany), and is
licensed under the terms of the Common Public License, Version 1.0 or (at
your option) any subsequent version.

The license is approved by the Open Source Initiative, and is
available from their website at \url{http://www.opensource.org}.

\bibliographystyle{alpha}
%\bibliography{/home/paper/biblio/paper.bib}


\end{document}
