%\documentclass[11pt]{article}
\documentclass[11pt,a4paper,halfparskip]{scrartcl}
%\usepackage[pdftex]{graphics} 
\usepackage{graphicx}
\usepackage[utf8x]{inputenc}
\usepackage{url} 
\usepackage[T1]{fontenc}
\usepackage{ucs} 

%\setkomafont{sectioning}{\bfseries}
\pagestyle{plain}
%\typearea{10}

\title{\small{Documentation for}\\\huge UIMA Wrapper for JULIE Lab Named Entity Tagger\\\vspace{3mm}\small{Version 2.0.0}}

\author{\normalsize Katrin Tomanek ~~~~~~~~~~~~~ Erik F\"assler\\
  \normalsize  Jena University Language \& Information Engineering (JULIE) Lab\\
  \normalsize F\"urstengraben 30 \\
  \normalsize D-07743 Jena, Germany\\
  {\normalsize \tt katrin.tomanek@uni-jena.de} }

\date{}

\begin{document}
\maketitle

\section{Objective}

The UIMA Wrapper for JULIE Lab Named Entity Tagger (UIMA-JNET) is an
UIMA wrapper for the JULIE Lab named entity tagger (JNET).  It is part
of the JULIE Lab JCORE NLP tool suite\footnote{\url{http://www.julielab.de/}}
which contains several NLP components (all UIMA compliant) from
sentence splitting to named entity recognition and normalization as
well as a comprehensive UIMA type system.

For detecting and classifing named entity mentions in a document, this
analysis engine employs the JULIE Lab Named Entity Tagger (JNET). JNET
is a named entity tagger that uses a machine learning (ML) approach.
It generates (ML-)features in order to recognise named entities in a
given text of written natural language.  JNET offers the possibility
to configure the feature generation and allows to use arbitrary
available annotations, so-called meta data, for its features. As JNET
does the named entity recognition, the UIMA-JNET provides all needed
data. It takes sentence and token annotations from its CAS as well as
meta data potentially required by JNET. It then writes the found named
entity mentions and optionally a given corresponding resource ID back
to the CAS.


\section{About this documentation}
This is a documentation on using the UIMA-compliant version of
JNET. UIMA-JNET is a wrapper to JNET, which actually does all the
named entity recognition. To get more information on JNET itself,
please refer to its documentation.


%%\section{Changelog}
%
%\subsection*{For Version 2.3}
%\begin{itemize}
%\item negative list can now be specified: entries in this list will
%  never be recognized as entities
%\item UIMA-JNET now writes to the type
%  \url{de.julielab.jules.types.EntityMention}. This is due to type
%  system changes. In earlier versions of UIMA-JNET, the type
%  \url{Entity} was used instead. Now, a feature called
%  \url{textualRepresentation} in \url{EntityMention} is also filled by
%  UIMA-JNET.
%\item consistency preservation mode (based on strings and
%  abbreviations) now available
%\item abbreviation expansion improved, bug fixed
%\item several methods rewritten to clean up the code
%\end{itemize}
%% uncomment when needed


\section{Requirements and Dependencies}

UIMA-JNET is completely written in Java using Apache UIMA\footnote{\url{https://uima.apache.org}}. It
requires Java 1.7 (or above).

The input and output of an AE takes place by annotation objects. The
classes corresponding to these objects are part of a \emph{JULIE Lab
  UIMA Type System}.\footnote{The \emph{JULIE UIMA type system} can be
  obtained separately from \url{http://www.julielab.de/}.} When refering to UIMA annotation types we mean types from
the JULIE Lab UIMA type system. 



\subsection{JNET}

UIMA-JNET is based on JNET which employs the machine learning toolkit
MALLET \cite{McCallum2002}. JNET does the named entity recognition. It
receives the text to be tagged sentence-by-sentence. On a token-level,
each token is assigned a class by the tagger which is either one of
the entity classes or, in case that this token is not part of an
entity mention, the common \texttt{OUTSIDE} label ``O''.

For recognizing named entities, JNET generates a set of machine
learning features. It is possible to provide already available
annotations of the text to be tagged. These so-called \textit{meta
  datas}, e.g. PoS information, are employed for feature
generation. It may be configured which meta data is supposed to be
used. This is done by a \textit{configuration file} that may be passed
to JNET in any training process. The file consists of key-value pairs
whereas the key is a setting-variable and the value its concrete
setting. The keys refering to meta data settings are of the form
\textit{``xxx\_feat\_data''}. ``xxx'' stands for the meta data name
and the rest of the string for the setting-variable which receives a
value. For details on this topic please see the JNET
documentation\footnote{JNET and its documentation can be obtained
  separately from \url{http://www.julielab.de/}}. In the following
tabular the configuration settings only interesting in relation to the
UIMA UIMA-JNET are explained. Please refer to the JNET documentation
for more information.

\begin{table}[h!]
\begin{tabular}{|l|l|p{6cm}|l|}
\hline
\textsc{key} & \textsc{allowed values} & \textsc{description}\\
\hline\hline

xxx\_feat\_data & string &  a string representing the path to the corresponding
UIMA-typesystem class for this annotation\\

\hline

xxx\_feat\_valMethod & string & the name of the method of the class referenced
to in "xxx\_feat\_data" for getting the value of the annotation (without
parameter brackets)\\

\hline

xxx\_begin\_flag & true / false & indicates if an IOB-like begin flag should be
used; useful for annotations spanning multiple tokens\\
\hline

\end{tabular}
\caption{Meta Data Feature Settings of JNET which are related to the UIMA
UIMA-JNET}
\end{table}


\subsection{Model}
UIMA-JNET needs a model previously trained. For training you need to
use JNET directly; just type the command:
\begin{verbatim}
runJNET.sh jar t
\end{verbatim}
,whereas the \textit{´t'} at the end of the string indicates a training
process, will result in the following output:
\begin{verbatim}
usage: runJNET.sh jar t <trainData.ppd> <tags.def> <model-out-file>
[featureConfigFile]
\end{verbatim}
Training requires training data in \emph{piped format (PPD)}
(\textit{<trainData.ppd>}), the tagset (\emph{<tags.def>)} and the
feature model file name (\emph{<model-out-file>}).  Optionally, you
may pass a feature configuration file. The output of a training
process is a model which may be used for prediction.
For details about the piped format, tagsets and feature configuration files
please refer to the JNET documentation.

\subsection{Connections Between Model Training and Running UIMA-JNET}
The machine learning features\footnote{on ``features'' is refered in the
meaning of a generic feature predicate rather than a concrete feature instance}
created during a training process are also generated during the prediction
process. This is done on the training text or the text to be predicted,
respectively. A model contains the storage of ML-features builded while
training. Using a model for prediction requires to generate the features
that were used for the creation of the model. Therefore it is necessary to
provide the same information. The meta datas used in the
training process are required to be available in the prediction process. That
is, if PoS information were used for the training process, a PoS annotation of
the text to be predicted must also be provided. In the UIMA environment the CAS
on which the UIMA-JNET refers needs to have the meta information available that
were used during the training process in order to work properly.


\section{Using the AE -- Descriptor Configuration}

In UIMA, each component is configured by a descriptor in XML. In the
following we describe how the descriptor required by this AE can be
created (or modified) with the \emph{Component Descriptor Editor}, an
Eclipse plugin which is part of the UIMA SDK.

A descriptor contains information on different aspects. The following
subsection refers to each sub aspect of the descriptor which is, in
the Component Descriptor Editor, a separate \emph{tabbed page}. For an
indepth description of the respective configuration aspects or tabs,
please refer to the \emph{UIMA SKD User's
  Guide}\footnote{\url{http://incubator.apache.org/uima/}}, especially
the chapter on ``Component Descriptor Editor User's Guide''.

To define your descriptor go through each tabbed pages mentioned here,
make your respective entries (especially in page \emph{Parameter
  Settings} you will be able to configure UIMA-JNET to your needs) and
save the descriptor as \url{SomeName.xml}.

\paragraph{Overview}
This tab provides general informtion about the component. For the UIMA-JNET
you need to provide the information as specified in Table
\ref{tab:overview}.

\begin{table}[h!]
  \centering
  \begin{tabular}{|p{3.5cm}|p{4cm}|p{6cm}|}
    \hline
    Subsection & Key & Value \\
    \hline\hline
    Implementation Details & Implementation Language &  Java\\
    \cline{2-3}
    & Engine Type & primitive\\
    \hline
    Runtime Information & updates the CAS & check \\
    \cline{2-3}
    & multiple deployment allowed & check\\
    \cline{2-3}
    & outputs new CASes & don't check \\
    \cline{2-3}
    & Name of the Java class file & \url{de.julielab.jules.ae.netagger.EntityAnnotator}\\
    \hline
    Overall Identification Information & Name & UIMA-JNET \\
    \cline{2-3}
    & Version &  2.3\\
    \cline{2-3}
    & Vendor & JULIE Lab\\
    \cline{2-3}
    & Description & you may keep this empty\\
    \hline
  \end{tabular}
  \caption{Overview/General Settings for AE.}
  \label{tab:overview}
\end{table}


\paragraph{Aggregate}
Not needed here, as this AE is a primitive.

\paragraph{Parameters}
\label{sss:parameters}

See Table \ref{tab:parameters} for a specification of the
configuration parameters of this AE. Do not check ``Use Parameter
Groups'' in this tab.

\begin{table}[h!]
  \centering
  \begin{tabular}{|p{4cm}|p{2cm}|p{2cm}|p{2cm}|p{4cm}|}
    \hline
    Parameter Name & Parameter Type & Mandatory & Multivalued & Description \\
    \hline\hline
    ModelFilename & String & yes & no & specifies which model JNET
    should use\\
    \hline
    EntityTypes & String & yes & yes & specifies which JNET named entity
    label should be represented by which typesystem class within UIMA\\
    \hline
    ExpandAbbreviations & Boolean & no & no & if set to true then abbreviations (if annotated in the CAS) are expanded by their full form. Entity recognition is then performed on the full form (instead of the short form).\\
    \hline
    ShowSegmentConfidence & Boolean & no & no & if set to true, a the classifier's confidence on each entity is calculated. See Section~\ref{sss:segconf} for more details.\\
    \hline
    NegativeList & String & no & no & a list with entity mentions (covered text) and label which, when predicted as entity, will be ignored, i.e., is not written to the CAS. \\
    \hline
    ConsistencyPreservation & String & no & no & coma-separated list of modes. Keep blanc if consistency preservation should not be used. See Section \ref{sss:consistencypres} for more details.\\
    \hline  
  \end{tabular}
  \caption{Parameters of this AE.}
  \label{tab:parameters}
\end{table}

\paragraph{Parameter Settings}
\label{sss:param_settings}

The specific parameter settings are filled in here. For each of the
parameters defined in \ref{sss:parameters}, add the respective values
here (has to be done at least for each parameter that is defined as
mandatory). See Table \ref{tab:param_settings} for the respective
parameter settings of this AE.

\begin{table}[h!]
  \centering
  \begin{tabular}{|p{4cm}|p{4cm}|p{8cm}|}
    \hline
    Parameter Name & Parameter Syntax & Example \\
    \hline\hline
    ModelFilename & just give the complete path to the model file
    & \url{/my/path/to/genemodel.mod.gz} \\
    \hline
    EntityTypes & given by pairs in the form of ``<JNET label>=<path typesystem
    class>'' &
    ``gene-dna=de.julielab.jules.types.EntityMention''; note that although you might specifiy different types here, all types must be subtypes of \url{EntityMention}.\\
    \hline
    ExpandAbbreviations & boolean & see above\\
    \hline
    ShowSegmentConfidence & Boolean & see above\\
    \hline
    NegativeList & one entry per line: ``entity mention@label''. You might omit the label (if so then the this is a negative entry for an arbitry entity label), however, in this case you should have an @ at the end of the line!
    & IL-2@gene-protein\newline
    IL-2 receptor\newline
    HDA1@gene-generic\newline
    \\
    \hline
    ConsistencyPreservation & String & "string, full2acro"\\
    \hline
  \end{tabular}
  \caption{Parameter settings of this AE.}
  \label{tab:param_settings}
\end{table}

\paragraph{Type System}
\label{sss:type_system}
On this page, go to \emph{Imported Type} and add the following layers
of the \emph{JULIE UIMA Type System} (Use ``Import by Location''):
\url{julie-basic-types.xml}, \url{julie-morpho-syntax-types.xml}, and
\url{julie-semantics-types.xml}. In case you use special subtypes of
\url{EntityMention}, of course, this part of the type system needs to
be added as well.

\paragraph{Capabilities}
\label{sss:capabilities}
UIMA-JNET needs as input annotations from type
\url{de.julielab.jules.types.Sentence} and
\url{de.julielab.jules.types.Token}. It returns annotations from type
\url{de.julielab.jules.types.EntityMention}. As output type you might
specify any subtype of \url{EntityMention}. If you do so, do not
forget to add the respective type system defining this type.  See
Table \ref{tab:capabilities}.
% adapt if needed
\begin{table}[h!]
  \centering
  \begin{tabular}{|p{5cm}|p{2cm}|p{2cm}|}
    \hline
    Type & Input & Output \\
    \hline\hline
     de.julielab.jules.types.Sentence &  $\surd$ & \\
      \hline
     de.julielab.jules.types.Token &  $\surd$ & \\
      \hline
     de.julielab.jules.types.EntityMention & &  $\surd$  \\
      \hline
  \end{tabular}
  \caption{Capabilities of this AE.}
  \label{tab:capabilities}
\end{table} 



\paragraph{Index}

Nothing needs to be done here.

\paragraph{Resources}

Nothing needs to be done here.


\section{Further Explanation of some Functionalities of UIMA-JNET}

This section explains some of UIMA-JNET's features in more detail. 

\subsection{Segment Confidence}
\label{sss:segconf}
The calculation of the segment confidence follows the approach
proposed in \cite{Culotta2004}). The segment confidence is written to
the feature \url{confidence} of the annotation type (or any subtypes
of) \url{EntityMention}.

Note: entity-level confidence calculation might seriously slow down
UIMA-JNET. Thus, for processing large amounts of documents we advice
not to use this feature.

\subsection{Consistency Preservation}
\label{sss:consistencypres}
The idea of this feature is that within one document the same string
should recieve the same entity label (if at all). As UIMA-JNET does
the tagging on sentence-level, it might happen that within one
sentence as certain token is tagged as an entity whereas in another
sentence it is not. This is often the case when an entity is used in
an ``untypical'' context (i.e., it is surrounded by words very
different from the training material). To avoid inconsistent
annotation, the consistency preservation mode makes sure that when a
token (or a multi-token expression) is once recognized as an entity
mention, all other occurences of this token within an document are
also labeled as the same entity mention.

Further, if an acronym was detected as an entity, but the respective
full-form was not (or vice versa)\footnote{For this feature to work
  you need to run the acronym tagger beforehand!}, the respective
\url{EntityMention} annotation is copied.

This feature is currently under construction and will have to be
improved. Use it carefully!

Available modes are:
\begin{itemize}
\item string
\item full2acro
\item acro2full
\end{itemize}




%\section{Modifying the Descriptors}
%
%This PEAR package contain one descriptor for UIMA-JNET configured so
%that UIMA-JNET detects gene entities. Most probably, you do not intend
%to use UIMA-JNET for gene recognition but for any other entites
%types. To do so, train a model with JNET for the respective entity
%types and modify the descriptor in this package respectively (add youŕ
%model and define the EntityTypes mappings from the predicted labels to
%the UIMA types where the outputs should be written to).
%
%The model contained in this package was trained on the gene
%annotations of the\textsc{PennBioIE}
%corpus\footnote{http://bioie.ldc.upenn.edu/} corpus. You will find the
%training data (in the format JNET expects it and) we used to build
%this model in the directory \url{resources}. Using 10-fold
%cross-validation we achieved an F-score around 83\% on this data with
%JNET.
%
%\section{Copyright and License}
%
%This software is Copyright (C) 2008 Jena University Language \& Information
%Engineering Lab (Friedrich-Schiller University Jena, Germany), and is
%licensed under the terms of the Common Public License, Version 1.0 or (at
%your option) any subsequent version.
%
%The license is approved by the Open Source Initiative, and is
%available from their website at \url{http://www.opensource.org}.

\bibliographystyle{alpha}
\bibliography{literature}


\end{document}
